% main.tex
\documentclass{ecustproposal}

% ==================== 元文件信息 ====================
\title{基于神经群体模型的大脑皮层下-皮层动态的动力学研究}
\author{言书懿,杜老师*}
\date{}

% ==================== 封面信息设置 ====================
\ecustlogo{figures/ECUST.jpg} % 校徽图片路径
\reporttitle{硕士学位论文开题报告} % 报告标题
\studentname{言书懿} % 学生姓名
\studentid{12345678} % 学号
\college{数学学院} % 学院
\advisor{杜老师} % 导师
\thesistitle{基于神经群体模型的大脑皮层下-皮层} % 论文题目
\thesissubtitle{动态的动力学研究} % 副标题,标题过长也可用。注释该行,则该行不显示
\proposaltime{2025年10月} % 开题时间

% ==================== 页眉信息设置 ====================
\headertitle{基于神经群体模型的大脑皮层下-皮层动态的动力学研究}

\begin{document}
	
	% ==================== 封面 ====================
	\makecover
	
	% ==================== 目录 ====================
	\tableofcontents
	
	\newpage
	% ==================== 中摘 ====================
	\chinesetitle{基于神经群体模型的大脑皮层下-皮层动态的动力学研究}
	\chineseauthor{言书懿,杜老师*}
	\chineseaffiliation{华东理工大学\hspace{0.5em}数学学院}
	% 添加目录
	\phantomsection
	\addcontentsline{toc}{section}{摘要}
	% sections/abstract.tex

\begin{abstract}% 此处不与下面内容空行即可内容与“摘要”字样同行
	卒中是一种急性脑血管疾病。
	
	\keywords{神经群体模型;卒中}
	
\end{abstract}
	\newpage
	
	% ==================== 英摘 ====================
	\englishtitle{Dynamic Study on Subcortical-Cortical Dynamics Based on Neural Mass Models}
	\englishauthor{YAN Shuyi, DU Laoshi*}
	\englishaffiliation{School of Mathematics, East China University of Science and Technology}
	% 添加目录
	\phantomsection
	\addcontentsline{toc}{section}{Abstract}
	% sections/abstract_en.tex

\begin{abstracten}
	Stroke is an acute cerebrovascular disease.
	
	\enkeywords{neural mass model; stroke}
	
\end{abstracten}
	\newpage
	
	% ==================== 章节 ====================
	% sections/background.tex

\section{研究背景}

卒中(脑卒中,俗称中风)是一种急性脑血管疾病,由于脑部血管阻塞(缺血性卒中)或破裂(出血性卒中)导致脑组织缺血缺氧,进而引发神经功能缺损。在非传染性疾病 (NCD) 中,中风仍然是世界第二大死因,也是死亡和残疾总和的第三大死因\cite{valery2025},患者常出现运动、语言和认知障碍。
	\newpage
	% sections/review.tex

\section{文献综述}

神经群体模型(Neural Mass Models, NMMs)是研究大脑动力学的重要工具,旨在通过简化微分方程模拟神经种群的平均活动。Jansen和Rit模型(JR模型)作为经典NMM,为后续研究奠定了基础;而Wendling模型则通过扩展JR模型,增强了模拟复杂动力学(如癫痫活动)的能力。

\subsection{Jansen-Rit模型}

Jansen-Rit模型由Jansen和Rit于1995年提出,旨在模拟视觉皮层的诱发电位和α节律。

\subsubsection{模型结构}

模型的核心是两个变换:一是将平均脉冲密度转换为平均突触后电位(PSP)的线性传递函数,二是将膜电位转换为脉冲密度的静态非线性函数(S型曲线)。
	\newpage
	% sections/objective.tex

\section{研究目标}

全脑网络模型(WBNM)是理解大脑功能机制的重要计算框架,它通过耦合多个神经群体模型(NMMs)来模拟大规模脑区活动。

\subsection{模型构建与节点}

基于标准脑图谱(如Desikan-Killiany atlas)。
	\newpage
	% sections/contents.tex

\section{研究内容}

本章节详细阐述本研究的具体内容、方法、步骤和分析技术。

\subsection{模型设计与节点定义}

本研究将采用Wendling模型作为每个脑区的动力学框架。
	\newpage
	% sections/roadmap.tex

\section{技术路线与可行性}

\subsection{技术路线}

本研究的技术路线如下所示,展示了从数据准备到结果验证的完整流程:

\subsection{可行性分析}

模型基础成熟​:Wendling模型经过20余年发展,已成功应用于癫痫等疾病研究,其数学框架稳定可靠

	\newpage
	% sections/innovation.tex

\section{创新性与预期成果}

\subsection{创新性}

为疾病机制提供新见解

\subsection{预期输出}

发表高水平论文1-2篇,贡献于计算神经科学和临床神经学。
	\newpage
	% sections/basic.tex

\section{工作基础}

\subsection{前期准备}

本研究已完成部分前期工作,为项目奠定了基础。

\subsection{不足与问题}
​
项目仍处于初级阶段,存在较多问题。
	\newpage
	% sections/scheduling.tex

\section{研究进度安排}

\subsection{总体时间安排}
\begin{itemize}
	\item \textbf{研究周期:} 2025年9月-2026年6月(共3个季度)
	\item \textbf{初稿完成:} 2026年6月(第3季度末)
	\item \textbf{论文投稿:} 2026年6月-2026年12月
	\item \textbf{论文接收:} 2027年初
\end{itemize}

\subsection{近期季度详细安排}

% 添加时间线图表
\begin{figure}[htbp]
	\centering
	\begin{tikzpicture}[scale=0.8, transform shape]
		% 时间轴
		\draw[->, line width=1.5pt] (0,0) -- (12,0);
		\foreach \x/\month in {0/9月, 3/12月, 6/3月, 9/6月}
		\draw (\x,0.2) -- (\x,-0.2) node[below] {\month};
		
		% 里程碑
		\node[draw, fill=blue!20, align=center] at (0.5,1) {模型代码\\完成};
		\node[draw, fill=green!20, align=center] at (3.5,1.5) {大规模仿真\\完成};
		\node[draw, fill=red!20, align=center] at (6.5,1) {分析工作\\完成};
		\node[draw, fill=purple!20, align=center] at (9.5,1.5) {论文初稿\\完成};
		
		% 季度标注
		\node[rotate=90, anchor=south] at (1.5, -1) {第1季度};
		\node[rotate=90, anchor=south] at (4.5, -1) {第2季度};
		\node[rotate=90, anchor=south] at (7.5, -1) {第3季度};
		\node[rotate=90, anchor=south] at (10.5, -1) {第4季度};
	\end{tikzpicture}
	\caption{研究进度时间线}
\end{figure}
	\clearpage
	
	% ==================== 参考文献 ====================
	% 添加到目录但不编号
	\phantomsection
	\addcontentsline{toc}{section}{参考文献}
	\section*{参考文献}
	% 参考文献内容
	\printbibliography[heading=none] % 输出参考文献,不显示默认标题

\end{document}