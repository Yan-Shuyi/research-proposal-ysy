% sections/scheduling.tex

\section{研究进度安排}

\subsection{总体时间安排}
\begin{itemize}
	\item \textbf{研究周期:} 2025年9月-2026年6月(共3个季度)
	\item \textbf{初稿完成:} 2026年6月(第3季度末)
	\item \textbf{论文投稿:} 2026年6月-2026年12月
	\item \textbf{论文接收:} 2027年初
\end{itemize}

\subsection{近期季度详细安排}

% 添加时间线图表
\begin{figure}[htbp]
	\centering
	\begin{tikzpicture}[scale=0.8, transform shape]
		% 时间轴
		\draw[->, line width=1.5pt] (0,0) -- (12,0);
		\foreach \x/\month in {0/9月, 3/12月, 6/3月, 9/6月}
		\draw (\x,0.2) -- (\x,-0.2) node[below] {\month};
		
		% 里程碑
		\node[draw, fill=blue!20, align=center] at (0.5,1) {模型代码\\完成};
		\node[draw, fill=green!20, align=center] at (3.5,1.5) {大规模仿真\\完成};
		\node[draw, fill=red!20, align=center] at (6.5,1) {分析工作\\完成};
		\node[draw, fill=purple!20, align=center] at (9.5,1.5) {论文初稿\\完成};
		
		% 季度标注
		\node[rotate=90, anchor=south] at (1.5, -1) {第1季度};
		\node[rotate=90, anchor=south] at (4.5, -1) {第2季度};
		\node[rotate=90, anchor=south] at (7.5, -1) {第3季度};
		\node[rotate=90, anchor=south] at (10.5, -1) {第4季度};
	\end{tikzpicture}
	\caption{研究进度时间线}
\end{figure}